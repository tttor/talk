\section{Surrogate Loss Function}
\frame{\tableofcontents[currentsection, hideothersubsections]}

\begin{frame}
\frametitle{Surrogate Loss Fn: Intro}

WHAT:\\
handle some nonconvex problems
by minimizing “surrogate” loss functions that are convex
\vspace{2mm}

WHY:\\
the natural loss function is not convex
\vspace{2mm}

HOW:\\
to upper bound the nonconvex loss function by a convex surrogate loss function.
the requirements from a convex surrogate loss are as follows:
\begin{itemize}
    \item It should be convex.
    \item It should upper bound the original loss.
\end{itemize}
\end{frame}

\begin{frame}
\frametitle{Surrogate Loss Fn: Example}
example:
learning the hypothesis class of half-
spaces with respect to the 0 − 1 loss.

For example, in the context of learning halfspaces, we can define the so-called
hinge loss as a convex surrogate for the 0 − 1 loss, as follows:

\end{frame}



